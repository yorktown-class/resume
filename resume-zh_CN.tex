% !TEX TS-program = xelatex
% !TEX encoding = UTF-8 Unicode
% !Mode:: "TeX:UTF-8"

\PassOptionsToPackage{colorlinks,urlcolor=blue}{hyperref}


\documentclass{resume}


% \usepackage[colorlinks,urlcolor=blue,dvipdfm]{hyperref}

\usepackage{zh_CN-Adobefonts_external} % Simplified Chinese Support using external fonts (./fonts/zh_CN-Adobe/)
% \usepackage{zh_CN-Adobefonts_internal} % Simplified Chinese Support using system fonts
% \usepackage{linespacing_fix} % disable extra space before next section

\usepackage{cite}

\begin{document}

    \pagenumbering{gobble} % suppress displaying page number
    
    \name{杨弘}
    
    \basicInfo{
      \email{yanghong1484944458@gmail.com} 
      \textperiodcentered\ 
      \phone{(+86) 15058795800} 
    %   \textperiodcentered\ 
    %   \linkedin[billryan8]{https://www.linkedin.com/in/billryan8}
    }
    
    % \section{\faHeart\ 意向岗位}
    % 研发工程师 - 工程/算法
     
    \section{\faGraduationCap\  教育背景}
    
        \datedsubsection{\textbf{华东师范大学}, 上海}{2019 -- 至今}
        \textit{本科}\ 数据科学与工程, 预计 2023 年毕业
    
    
    \section{\faStar\ 竞赛经历}
        ICPC区域赛两金两银,EC final银牌。
        CCPC区域赛两金一银,CCPC final金牌。
        
        \subsection{ICPC}
            \begin{itemize}
                \item The 2020 ICPC Asia-East Continent Final Contest, Silver Medal
                \item The 2020 ICPC Asia Shanghai Regional Contest, Gold Medal
                \item The 2020 ICPC Asia YinChuan Regional Contest, Gold Medal
            \end{itemize}
        \subsection{CCPC}
            \begin{itemize}
                \item 2020 China Collegiate Programming Contest Finals, Rank 7, Gold Medal
                \item 2020 China Collegiate Programming Contest, Changchun Site, Gold Medal
                \item 2021 China Collegiate Programming Contest, Guilin Site, Rank 6, Gold Medal
            \end{itemize}
    
    \section{\faUsers\ 实习经历}
        \subsection{图森未来} 
            \datedline{工程部高精地图实习生}{2022.7.11 - 2022.11.25}
            地图SDK的维护,地图生产产线组件开发等等。使用的语言为C++和Python。
            
    \subsection{\faUsers\ 实习项目经历}
        \paragraph{TSMap} 图森高精地图C++ SDK\\
            feature: 实现多点批量查询曲率的API,并优化批量查询的效率(当批量查询点数小于曲线段数时,总消耗时间与单点查询大致相同)。\\
            此外发现原有曲率查询API存在的bug。
            
        \paragraph{MapBuilder} 从采集车获取的kml文件构建高精地图的C++项目\\
            feature: 处理了部分点点相等判断的问题\\
            一个点有三个浮点数表示。
            浮点数相等的判断存在两种逻辑:二进制相等和有eps误差的相等。
            原来的代码对等号的使用逻辑上比较混乱,部分流程允许误差、部分流程不允许误差。
            导致不允许误差的流程因为前面积累的误差而出现错误。
            
            分析了误差产生累积的原因后在大部分地方修复了问题。
        % \subsection{Format Convert Service, Post Processing Service: 地图生产产线格式转换服务,制图后处理服务}
        %     \begin{description}
        %         \item[格式转换服务] 主要功能是将采集到并经过质量检验的kml文件,转换为符合规格的json文件
        %         \item[制图后处理服务] 从已经经过质检的地图数据中自动生成冗余数据
        %     \end{description}
        \paragraph{DrivableMap} 渲染二维地图可驾驶区域的C++项目\\
            将二维地图按一定比例缩放到二维矩阵上。
            通过矩阵对应位置的值来表示相应区域是否有可驾驶地图元素。
            用于快速判断点是否在可驾驶区域内。

        \paragraph{地图数据生产pipeline} 将地图生产的任务拆分为各个节点,方便溯源和版本管理\\
            我完成了地图数据产线的部分节点的开发,并通过pytest进行单元测试, 最后以docker image的形式串联到整个流程中。
            \begin{itemize}
                \item 格式转换节点: 将数据从kml格式转换为json格式
                \item 制图后处理节点: 计算地图元素的冗余信息
            \end{itemize}

    \section{\faBook\ 个人项目}
        \paragraph{ \href{https://github.com/ya-hong/tiny_coroutine}{tiny\_coroutine}} 基于C++20的单线程并发协程框架\\
            对C++20的协程进行了封装,可以比较轻松的使用协程。同时实现了协程的时间片轮转,可以让协程在单线程上并发。
            
        \paragraph{ \href{https://github.com/ya-hong/recycled_ptr}{recycled\_ptr} } 有垃圾回收能力的指针\\
            实现垃圾回收指针。在已经指定gc root的前提下,通过标记清除法实现垃圾回收。

        \paragraph{ \href{https://github.com/ya-hong/SATranscriber}{SATranscriber} } 基于openai whisper的实时语音转文字工具\\
            在whisper的基础上实现对音频流的转录。在延迟2-3s的情况下,可以生成确定性的文字结果,方便接入翻译等下游任务。
    
    \section{\faCogs\ IT 技能}
    \begin{itemize}
      \item 编程语言: C++, Python
      \item 平台: Linux
      \item 其他: Git
    \end{itemize}

\end{document}

