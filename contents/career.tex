\section{\faUsers\ WORK EXPERIENCE}


\subsection{TUSIMPLE}
    \datedline{Map Engine Engineer}{2023.07.03 - 2024.03.12}
    \datedline{Map Intern}{2022.07.11 - 2022.11.25}
    \paragraph{Map SDK \textit{TSMap}}
        \begin{itemize}
            \item 
                TSMap is a C++ map SDK used by vehicle software in TuSimple.
                It is used by over 100 packages.
                % It efficiently provides the required map data and information 
                The SDK runs on the limited computing resources available onboard vehicles
                and provides required data in efficiency.
                It has passed more than a thousand drive-out tests during my maintenance.
            \item 
                Developed the Perception Drivable Map feature.
                It segments a 2D map into drivable and undrivable areas, 
                presented as a matrix.
                This matrix helps perception algorithms efficiently determine whether a point is within the drivable area.
            \item 
                Developed the Planning Drivable Map feature.
                It efficiently calculates the length of contiguous drivable areas in physical on both sides.
                It can process 5000 queries in 1000 meters within 3 ms and can perform calculations asynchronously.
        \end{itemize}

    \paragraph{\textit{Map Pipeline}}
        \begin{itemize}
            \item 
                Map Pipeline is a microservices system for map building
                that consumes raw sensor data, and produces map data.
                I am responsible for the development of a series of services that are
                human intervention-free and stateless.
                These services support processing input vector data 
                and outputting 
                basedata including lanes, intersections, topology, phiscal element relations.
            \item 
                Developed major services in the procedure that input vector data and output well-structured map data.
                They support updating map data with differential content.
                Used C++ for computationally heavy modules
                and Python for communication with other services.
                Used pybind to integrate the two parts.
                Used jenkins for CI/CD.
                Applied unit testing with GoogleTest and pytest.
            \item 
                Developed a few services to support
                converting data formats for data produced in other pipelines,
                filling missing fields in raw input,
                and merging divided map-building tasks.
        \end{itemize}

    \paragraph{Data compiler \textit{Great Map Builder}}
        \begin{itemize}
            \item 
                Great Map Builder is a map data compiler
                that consumes annotated vector data
                and produces binary protobuf files.
                These files includes lanes, intersections, topology, physical element relations, 
                and can be directly loaded by the map SDK.
            \item 
                I am responsible for the maintenance and development of Great Map Builder 
                as the primary developer after joining the team.
                I have developed features and fixed bugs in map building.
            \item 
                Extended the types of map elements.
                Added support for describing double line bounds, physical drivable areas, etc.
                Guranteed forward and backward compatibility.
        \end{itemize}

