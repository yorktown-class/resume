\section{\faUsers\ 工作经历}


\subsection{图森未来}
    \datedline{研发工程师——高精地图}{2023.07.03 - today}
    \datedline{工程部高精地图实习生}{2022.07.11 - 2022.11.25}
    \paragraph{高精地图 TSMap SDK 的维护和开发}
        \begin{itemize}
            \item 
                TSMap 是所有图森车上软件使用高精地图的 SDK, 总计被 100 多软件包所直接或间接依赖。
                TSMap 可以高效的为算法程序提供它们所需的地图数据和信息。
                正式入职后的大部分时间都是我在进行维护和 feature 开发,
                对路侧中出现的问题都进行了及时的修复。
                保障了路测稳定和算法使用。
            \item 
                根据自动驾驶感知算法的需求,开发 Perception Drivable Map 组件。
                能够快速绘制自车一定矩形范围内的表示可行驶区域的矩阵。
                一方面用于快速判断自车附近的物体是否在可行驶区域内,将计算量分布在在线的预处理阶段,更符合算法对性能的需求;
                另一方面可行驶区域矩阵可以方便的作为模型的输入参数。
            \item 
                根据自动驾驶运动规划算法的需求,开发 Planning Drivable Map 组件。
                能够快速计算路点法线上左右两侧的物理可连续行驶距离。
                在 Frenet 坐标系上,使用可行驶区域的公共边,通过 BFS 将路径拓展为封闭的左右连续可行驶的多边形,以此来支持快速在大量的纵向距离进行查询。
                开发完成了高效率的接口,能够用平均 3 ms 完成 1000 m 上 5000 个点的查询。
                同时允许异步更新并存储任意路线上的结果,以满足车上算法对性能的需求。
            \item 
                修复了一个由 fcntl 导致的多进程多线程程序假死锁问题。
                通过限制获取读锁的阻塞时间避免了错误的死锁检测。
        \end{itemize}

    \paragraph{Great Map Builder 的维护和开发}
        \begin{itemize}
            \item 
                Great Map Builder 是高精地图的数据编译软件。
                输入经过人工标注的矢量数据,自动构建出车道、路口、拓扑关系、物理元素关系等地图元素和属性,并导出可供软件直接使用的二进制 Protobuf 文件。
            \item 
                Great Map Builder 基本是我一人维护。完成修复了生产制图总遇到的各种 bug,保障了在洋山港、曹妃甸、澳洲、日本等地的建图。
            \item 
                根据算法和地图需求,拓展了地图元素类型,支持表达双线型边界、非车道可行驶区域等,同时保证了向前和向后兼容。
        \end{itemize}

    \paragraph{地图构建 pipeline}
        \begin{itemize}
            \item 
                地图构建 pipeline 是未来用来取代 Great Map Builder 等单体应用的微服务地图构建系统。
                满足公司未来对高精地图量产化的需求,能够持续稳定的交付规模化的数据。
                系统能达到可靠性高、产能稳定、成本可控。
            \item 
                我负责开发的主要为从矢量数据到地图母库数据,无人工参与,无状态的生产制图服务。
                使用 C++ 负责需要大量计算的数据处理部分,使用 Python 与其他服务交互,并通过 pybind 连接两种语言。
                通过 opentelemetry 接入可观测性平台,便于快速发现和溯源问题。
                通过 jenkins 实现 CI/CD。
                使用 GoogleTest 和 pytest 进行单测。
                完成负责的各服务的详细设计,使用合理算法高效构建地图元素和关系。
            \item 
                相比 Great Map Builder,构建了更丰富的地图母库数据,能够支持差分更新构建。
        \end{itemize}